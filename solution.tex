\documentclass{article}
\usepackage{graphicx}

\title{Advanced Information Management Assignment 1 }

\author{Markus Kaltenecker, Lukas Gillich, Axel Körner}

\begin{document}

\maketitle

\section{Task 1}

\subsection{}
The two most important rules for a well-formed XML-Document are the following:

\begin{enumerate}
\item The XML-Document has a single top-level element
\item Every start tag in the XML-Document has a unique matching end tag, that is in the context of the same parent element
\end{enumerate}
The given document is well-formed since it has only a single top-level element \texttt{<tns:Orders>}. Furthermore, every start tag has a matching end tag that is in the context of the same parent element.

\subsection{}
The following code lines have invalidities:
\begin{enumerate}
    \item Line 53: The \texttt{itemID} is the same as in line 48.
    \item Line 26: A quantity of 10 is not valid, since the schema prohibits the element to have a value of 10 or more.
    \item Line 44: The \texttt{ZipCode} has a 4-digit value but the schema enforces a 5-digit value.
    \item Line 41: The schema requires the attribute \texttt{addressID} for the tag \texttt{BillingAdress}, which is not present.

\end{enumerate}

\subsection{}
No, a valid XML document must be well-formed. The validity requires it to be well-formed.


\section{Task 2}

\subsection{}

\subsection{}

\subsection{}

\subsection{}

\section{Task 3}

\subsection{}

\subsection{}

\subsection{}

\end{document}
